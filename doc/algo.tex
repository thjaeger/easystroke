\documentclass{article}
\usepackage{amsfonts}
\newcommand{\R}{\mathbb{R}}
\begin{document}
A stroke $z: [0,1] \rightarrow \R^2$ is represented by a sequence of line
segments in such a way that $z'(t)$ is equal to the same constant almost
everywhere.  
\begin{enumerate}
\item
Define
\[ \tilde{d}(z,w) = \min_{a \in \R^2, k \in \R} \int_0^1 \|z(t) - k w(t) - a\|^2 \, dt \]
then the distance of $z$ and $w$ is given by
\[ d(z,w) = \pm \sqrt{1-\frac{\tilde{d}(z,w)}{\tilde{d}(z,0)}} \]
where the sign depends on whether the minimum is attained for a positive or
negative $k$.  Clearly, $|d(z,w)| \leq 1$.  It turns out that $d$ is symmetric,
but there is no ``triangle inequality'' that I am aware of.

\item
More generally, for $0 \leq p \leq 1$, we define
\[ d^k(z, w) = \min_{a \in \R^2} \int_0^1 \|z(t) - k w(t) - a\|^2 \, dt\]
\[ \tilde{d}^k(z, w) = \int_0^1 \|z'(t) - k w'(t)\|^2 \, dt\]
\[ d_p(z, w) = \pm \sqrt{1 - \min_{k \in \R}\left( (1-p)\frac{d^k(z,w)}{d^k(z,0)} + p\frac{\tilde{d}^k(z,w)}{\tilde{d}^k(z,0)} \right)} \]
It is easy to see that $|d_p(z,w)| \leq 1$ and $d_0(z,w) = d(z,w)$.
Unfortunately $d_p$ is in general not symmetric anymore, but usually
$d_p(z,w)$ and $d_p(w,z)$ are close.

\end{enumerate}

\end{document}
